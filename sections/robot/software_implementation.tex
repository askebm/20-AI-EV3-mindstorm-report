% !TEX program = lualatex
\documentclass[../../main.tex]{subfiles}
\begin{document}
\subsection{Software implementation}%
\label{sub:software_implementation}

\subsubsection{Solution interpretor}%
\label{ssub:solution_intepretor}

The solution format provided for the project is shown in Table~\ref{tab:solution_format}.
An example solution could be \texttt{ullldluRRRR}.
This solution is is converted to \texttt{ullldlurrrR}. This is to utilise the uppercase letter to
mean, push the can to the next cross and reverse back to previous cross.
This interpretor makes it so the robot is compliant with requirement \ref{ssub:map} while using
a custom internal format.

\begin{table}[h]
	\centering
	\caption{Solution format}
	\label{tab:solution_format}
	\begin{tabular}{*{5}{l}}
		\toprule
		Without pushing can & \tt u &\tt  l & \tt r & \tt d \\
		While pushing can & \tt U & \tt L & \tt R & \tt D \\
		\cmidrule{2-5}
											& Up & Left & Right & Down\\
											\bottomrule
	\end{tabular}
\end{table}


\subsubsection{Navigation}%
\label{ssub:navigation}

The main code of the robot can be seen in Listing~\ref{lst:nav}.
As can be seen the robot completes one action at the time.
This is to keep everything simple and robust
as robustness is a part of requirement \ref{ssub:precision}.
The only difference between an upper case action and a lower case action
is that the robot will push the can on to the next cross and reverse back to the previous
cross. This would not make sense to do at every time a can need to pushed if it is pushed in
the same direction, this is the reason the solution interpretor is implemented, as described in
section \ref{ssub:solution_intepretor}.

\begin{listing}
	\caption{Navigation code}	
	\label{lst:nav}
	\begin{minted}{python}
def run(self):
		for action in self.solution:
				self.change_dir(action.lower())
				self.follow_line(
								condition_func = follow_until_color,
								sensor1=self.cL,
								sensor2=self.cR )
				if action.isupper():
						self.tank.on_for_degrees(SpeedPercent(30),SpeedPercent(30),30)
						self.follow_line(
										condition_func=follow_for_distance,
										distance=470 )
						while self.cL.reflected_light_intensity  > self.c_thresh and \
								self.cL.reflected_light_intensity > self.c_thresh :
								self.tank.on(SpeedPercent(-30),SpeedPercent(-30))
						self.tank.on_for_degrees(SpeedPercent(30),SpeedPercent(30),20)
				self.tank.stop()
	\end{minted}
\end{listing}

\subsubsection{Line follower}%
\label{ssub:line_follower}

As the robot needs to follow a line between crosses as of requirement \ref{sub:sensing} and
\ref{ssub:map}, a line follower has been implemented.
It uses two color sensors to robustly follow the line and to detect
when an intersection has been reach.\\
The line follower turns with a constant velocity to not overshoot the yaw angle as
there is made no attempt to estimate where the line is spatially and the optimal robot
yaw is therefor unknown.\\
Detection of intersections is prioritised over line detection as the robot would otherwise
have undefined behavior at intersections.\\
The line follower is implemented such that it can be provided a condition function.
To condition function has been implemented; \mintinline{python}{follow_until_color} and
\mintinline{python}{follow_for_distance}.



	
\end{document}
