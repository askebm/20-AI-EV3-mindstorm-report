% !TEX program = lualatex
\documentclass[../../main.tex]{subfiles}
\begin{document}
\subsection{Software implementation}%
\label{sub:software_implementation}

\subsubsection{Solution interpretor}%
\label{ssub:solution_intepretor}

The solution format provided for the project is shown in Table~\ref{tab:solution_format}.
An example solution could be \texttt{ullldluRRRR}.
This solution is is converted to \texttt{ullldlurrrR}. This is to utilise the uppercase letter to
mean, push the can to the next cross and reverse back to previous cross.

\begin{table}[h]
	\centering
	\caption{Solution format}
	\label{tab:solution_format}
	\begin{tabular}{*{5}{l}}
		\toprule
		Without pushing can & \tt u &\tt  l & \tt r & \tt d \\
		While pushing can & \tt U & \tt L & \tt R & \tt D \\
		\cmidrule{2-5}
											& Up & Left & Right & Down\\
											\bottomrule
	\end{tabular}
\end{table}


\subsubsection{Navigation}%
\label{ssub:navigation}

An overview of how the robots navigates a solution can be seen below.
\begin{minted}{python}
for action in self.solution:
		change_direction(action)
		follow_line_until_cross()
		if action.isupper():
				follow_line_for_distance()
				reverse_until_see_cross()
stop()
\end{minted}
As can be seen, the focus of the robot code is to be simple and robust.








	
\end{document}
