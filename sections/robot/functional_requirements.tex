\documentclass[../../main.tex]{subfiles}
\begin{document}

This section is about the functional requirements of the robot so that it is able to understand its environment and act correctly in it.

\subsubsection{Sensing}%
\label{sub:sensing}

It is critical that the robot is able to detect the two black lines that denote a tile. This is the foundation of the robot's ability to collect feedback about where it is on the map.

\subsubsection{Map}%
\label{ssub:map}

The ability to accept input in the form of left, up, down and right, and because of this, is able to complete a sokoban map.
The robot should be able to detect either side of a line to evaluate whether or not the robot is following the lines on the map.

\subsubsection{Movement}%
\label{sub:movement}

The movement of the robot is another important aspect and will cover aspects such as the precision of the movement.\\
A fundamental requirement is that the robot must have the ability to move around on a sokoban map and follow the rules. It must have various qualities that enables it to complete the objective of sokoban.

\subsubsection{Precision}%
\label{ssub:precision}

The robot should be precise with its movement as recovering is impossible.
The ability to move with a can and somewhat precisely place it on a tile is also critical to be able to complete the objective of sokoban. It is also needed that the robot is able to capture a can with its inlet.

\subsubsection{Agility}%
\label{ssub:agility}

The robot should have the ability to turn and not interfere with a can or touch another tile adjacent to it. This is to ensure that the robot is adhering to the rules of sokoban. \\
When the robot turns after detecting a new tile it should be able to rotate and be positioned correctly so that it is able to follow the black line towards the next tile.
	
\end{document}