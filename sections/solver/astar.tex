% !TEX program = lualatex
\documentclass[../../maint.tex]{subfiles}
\begin{document}

\subsubsection{Representation}

In A* a graph representation is used instead of a tree and a dictionary is kept to keep track of what node the next one was accessed through.

\subsubsection{Heuristics}

As A* uses heuristics to choose what node to discover next there should be careful considerations as to how a node is 			getting evaluated. The goal is not simply to move the player to a specific location, but rather move multiple boxes in a particular way so they end up on a goal.
Manhatten distance has been used for the heuristic since sokoban is based on a grid and there is no diagonal movement. 			Manhatten distance is the total number of tiles moved horizontally and vertically to reach the target tile from the 			current tile. Another requirement is that the heuristic function must be admissible, which means it can never overestimate 	the cost to reach the goal. The reason for the admissible requirement is that it is critical to find the optimal path to the goal from the start position to reduce the time it takes for the robot to solve the sokoban level.

\subsubsection{Solver results}

 \begin{itemize}

	\item Runtime 		
 		
 	Tests were made on numerous levels to get a better understanding of the characteristics of the solver.
 		
 	level 55 has a time of 441 seconds
 		
	\item Memory
		
	As A* creates fewer states it will also use less memory. 
		
	level 55 has a memory consumption of 1.7 GB
		
 \end{itemize}
	
\end{document}
