% !TEX program = lualatex
\documentclass[../../main.tex]{subfiles}
\begin{document}

Breadth first mostly makes sense when the distance between nodes are equal.

This breadth first solution generates a new state for each player step.

The worst case states generated can be caluculated as:

\begin{equation}
	\textrm{empty squares}^{ \textrm{Number of boxes}}
\end{equation}

This infers that the number of boxes is the determining factor for the size
of the generated tree from breadth first search.

The worst case time complexity can be calculated as

\begin{equation}
	4^{  \left(  \textrm{Empty Squares} ^{ \textrm{Number of Boxes}}  \right)   }
\end{equation}

This will probably always be lower as it is not always possible to move in all 4 direction.


	
\end{document}
