%===================================================
%================= Geometry ========================
%===================================================
% link: https://www.ctan.org/pkg/geometry
% Description:
%		The package provides an easy and flexible user interface to customize page layout,
%		implementing auto-centering and auto-balancing mechanisms so that the users have only
%		to give the least description for the page layout.
% Note: Should be before any other page layout, such as adding rule in the header
\usepackage[headheight=14pt, top=1.5cm, bottom=1.5cm, left=1.0cm, right=4.5cm]{geometry}

%===================================================
%=============== Fontspec ==========================
%===================================================
% link: https://www.ctan.org/pkg/fontspec
% Description:
%		Fontspec is a package for XELATEX and LuaLATEX.
%		It provides an automatic and unified interface to feature-rich
%		AAT and OpenType fonts through the NFSS in LATEX running on XETEX or LuaTEX engines.
\usepackage{fontspec}

%===================================================
%=============== Polygloglossia ====================
%===================================================
% link: https://www.ctan.org/pkg/polyglossia
% Description:
% 	This package provides a complete Babel replacement for users of LuaLATEX and XELATEX;
%		it relies on the fontspec package, version 2.0 at least.
% Note: Used to change language of table of contents
\usepackage{polyglossia}
\setdefaultlanguage{english}

%===================================================
%================ Amsmath ==========================
%===================================================
% link: https://www.ctan.org/pkg/amsmath
% Description:
%		The principal package in the AMS-LATEX distribution.
%		It adapts for use in LATEX most of the mathematical features found in AMS-TEX;
%		it is highly recommended as an adjunct to serious mathematical typesetting in LATEX.

%		When amsmath is loaded, AMS-LATEX packages amsbsy (for bold symbols),
%		amsopn (for operator names) and amstext (for text embedded in mathematics) are also loaded.

%		amsmath is part of the LATEX required distribution; however, several contributed packages 
%		add still further to its appeal; examples are empheq, which provides functions for
%		decorating and highlighting mathematics, and ntheorem, for specifying theorem
%		(and similar) definitions.
\usepackage{amsmath}

%===================================================
%================ Subfiles =========================
%===================================================
% link: https://www.ctan.org/pkg/subfiles
% Description:
%		Using subfiles the user can handle multi-file projects more comfortably making it possible
%		to both process the subsidiary files by themselves and to process the main file that
%		includes them, without making any changes to either.
\usepackage{subfiles}


%===================================================
%================= todonotes =======================
%===================================================
% link: https://www.ctan.org/pkg/todonotes
% Description:
%		The package lets the user mark things to do later, in a simple and visually appealing way.
%		The package takes several options to enable customization/finetuning of the visual appearance.
\usepackage[textwidth=4cm]{todonotes}

